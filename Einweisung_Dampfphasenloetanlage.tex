%%%%%%%%%%%%%%%%%%%%%%%%%%%%%%%%%%%%%%%%%%%%%%%%
% COPYRIGHT: (C) 2012-now FAU FabLab and others, CC-BY-SA 3.0
%%%%%%%%%%%%%%%%%%%%%%%%%%%%%%%%%%%%%%%%%%%%%%%%


% Um zum Text zu gelangen, einfach herunterscrollen (bzw. nach dem gew{\"u}nschten Inhalt suchen.)

%%%%%%%%%%%%%%%%%%%%%%%%%%%%%%%%%%%%%%%%%%%%%%%%
% TeX-Pakete
%%%%%%%%%%%%%%%%%%%%%%%%%%%%%%%%%%%%%%%%%%%%%%%%
\newcommand{\basedir}{fablab-document}
\documentclass{\basedir/fablab-document}

% \usepackage{fancybox} %ovale Boxen f{\"u}r Kn{\"o}pfe - nicht mehr ben{\"o}tigt
\usepackage{amssymb} % Symbole f{\"u}r Kn{\"o}pfe
\usepackage{subfigure,caption}
\usepackage[ngerman]{cleveref} % \cref{label} f{\"u}r Verweise wie "Abbildung 6", "Kapitel 3.1"

\usetikzlibrary{shapes,arrows} % f{\"u}r das flowchart

\usepackage{marvosym} % f{\"u}r Briefumschlag-Symbol
\usepackage{eurosym}
\usepackage{tabularx} % Tabellen mit bestimmtem Breitenverh{\"a}ltnis der Spalten
\usepackage{multirow} % Tabellen Zellen die sich {\"u}ber mehrere Zeilen ausdehnen
\usepackage{wrapfig} % Textumlauf um Bilder

\renewcommand{\texteuro}{\euro}

\linespread{1.2}
\fancyfoot[L]{kontakt@fablab.fau.de}
\title{Einweisung Dampfphasenl{\"o}tanlage \textit{VaporPhase One}}

\tikzstyle{laserknopf} = [anchor=base, draw=black, fill=gray!10, rectangle, rounded corners, inner sep=2pt, outer sep = 3pt]
\tikzstyle{lueftungsknopf} = [anchor=base, color=white, draw=black, fill=gray, rectangle, rounded corners, inner sep=2pt, outer sep = 3pt]

% styles f{\"u}r das Flowchart
\tikzstyle{decision} = [diamond, draw, fill=blue!20,
text width=4.5em, text badly centered, node distance=3cm, inner sep=0pt]
\tikzstyle{block} = [rectangle, draw, fill=blue!20,
text width=5em, text centered, rounded corners, minimum height=4em]
\tikzstyle{line} = [draw, very thick, color=black!50, -latex']
\tikzstyle{cloud} = [draw, ellipse,fill=red!20, node distance=3cm,
minimum height=2em]

% Kn{\"o}pfe f{\"u}r Laser und Fernbedienung
\newcommand{\knopf}[2]{
	\begin{tikzpicture}[baseline={(box.base)}]
	\node [#1] (box) {
		\fontsize{9pt}{9pt}\selectfont \textbf{#2}\strut
	};
	\end{tikzpicture}
}

%%%%%%%%%%%%%%%%%%%%%%%%%%%%%%%%%%%%%%%%%%%%%%%%
% diverse Befehle
%%%%%%%%%%%%%%%%%%%%%%%%%%%%%%%%%%%%%%%%%%%%%%%%
\newcommand{\nurZing}{\emph{nur f{\"u}r Epilog Zing:} }
\newcommand{\nurLTT}{\emph{nur f{\"u}r LTT iLaser:} }
\renewcommand{\todo}[1]{\textbf{\color{red}{TODO: #1}}}
\newcommand{\pfeil}{\ensuremath{\rightarrow}}


%%%%%%%%%%%%%%%%%%%%%%%%%%%%%%%%%%%%%%%%%%%%%%%%
% Befehle f{\"u}r die Kn{\"o}pfe,
% z.B. \laserLTTStop f{\"u}r die Stop-Taste am LTT iLaser
%%%%%%%%%%%%%%%%%%%%%%%%%%%%%%%%%%%%%%%%%%%%%%%%
\newcommand{\laserKnopf}[1]{\knopf{laserknopf}{#1}}
\newcommand{\laserZingXyAus}{\laserKnopf{X/Y aus}}
\newcommand{\laserRoterLaser}{\laserKnopf{roter Laser}}
\newcommand{\laserZingStart}{\laserKnopf{Start}}
\newcommand{\laserZingStop}{\laserKnopf{Stop}}
\newcommand{\laserZingReset}{\laserKnopf{Reset}}
\newcommand{\laserLTTPause}{\laserKnopf{$\blacktriangleright\,\parallel$}} % TODO: sch{\"o}neres Pause-Symbol suchen
\newcommand{\laserLTTNextJob}{\laserKnopf{$\blacktriangleright\blacktriangleright$}}
\newcommand{\laserLTTPreviousJob}{\laserKnopf{$\blacktriangleleft\blacktriangleleft$}}
\newcommand{\laserLTTStop}{\laserKnopf{$\square$}}
\newcommand{\laserLTTExit}{\laserKnopf{$\curvearrowleft$}} % Pfeil linksrum
\newcommand{\laserLTTOkay}{\laserKnopf{$\checkmark$}}
\newcommand{\laserJob}{\laserKnopf{Job}}
\newcommand{\laserFocus}{\laserKnopf{Focus}}
\newcommand{\laserPfeilRauf}{\laserKnopf{$\blacktriangle$}}
\newcommand{\laserPfeilRunter}{\laserKnopf{$\blacktriangledown$}}
\newcommand{\laserLTTHome}{\laserKnopf{\includegraphics{img/icon-home.pdf}}}


\newcommand{\laserDisplay}[1]{\fbox{\texttt{#1}}}

\newcommand{\lueftungKnopf}[1]{\knopf{lueftungsknopf}{#1}}
%der echte Enterpfeil geht leider nicht: \newcommand{\lueftungEnter}{\lueftungKnopf{↲}}
% stattdessen ein gespiegelter Pfeil ↰, der fast genauso aussieht.
\newcommand{\reflectboxX}[1]{\raisebox{\depth}{\scalebox{1}[-1]{#1}}} % Spiegelung an x-Achse
\newcommand{\returnSymbol}{\reflectboxX{\ensuremath{\mathbf{\Lsh}}}} % Enterpfeil (sieht fast genauso aus)
\newcommand{\lueftungEnter}{\lueftungKnopf{\returnSymbol}}

%\newcommand{\lueftungEnter}{\lueftungKnopf{\ensuremath{\mathbf{\hookleftarrow}}}}
\newcommand{\lueftungMinus}{\lueftungKnopf{-}}
\newcommand{\lueftungPlus}{\lueftungKnopf{+}}
\newcommand{\lueftungOn}{\lueftungKnopf{On}}
\newcommand{\lueftungOff}{\lueftungKnopf{Off}}
\newcommand{\lueftungPfeilRauf}{\lueftungKnopf{$\blacktriangle$}}
\newcommand{\lueftungPfeilRunter}{\lueftungKnopf{$\blacktriangledown$}}


%VisiCut Buttons
\newcommand{\button}[1]{\knopf{lueftungsknopf}{#1}}

% English translation in gray:
% \english{text}
\makeatletter
% \smaller: switch to small fontsize if current font is > 10pt (i.e., not footnote)
\newcommand{\smaller}{\ifdim\f@size pt>10pt \small \fi}
\makeatother
\newcommand{\english}[1]{{\smaller \color{gray} \itshape #1}}

%% REMOVE THE FOLLOWING LINE TO SHOW THE ENGLISH TRANSLATIONS
\renewcommand{\english}[1]{}

% Reference to previous footnote, needed when translating footnotes.
% Usage:
%   Hallo Welt \footnote{\label{myLabel} Fu{\ss}note}
%   Hello World \footnoteref{myLabel}.
\newcommand{\footnoteref}[1]{\textsuperscript{\ref{#1}}}

%knöpfe für VaporPhase One
\newcommand{\vpoMount}{\laserKnopf{Mount}}
\newcommand{\vpoEject}{\laserKnopf{Eject}}
\newcommand{\vpoEco}{\laserKnopf{Eco}}
\newcommand{\vpoStandard}{\laserKnopf{Standard}}
\newcommand{\vpoFast}{\laserKnopf{Fast}}
\newcommand{\vpoSet}{\laserKnopf{Set}}
\newcommand{\vpoOpen}{\laserKnopf{Open Lid}}
\newcommand{\vpoClose}{\laserKnopf{Close Lid}}

%%%%%%%%%%%%%%%%%%%%%%%%%%%%%%%%%%%%%%%%%%%%%%%%
% Inhalt des Dokuments
%%%%%%%%%%%%%%%%%%%%%%%%%%%%%%%%%%%%%%%%%%%%%%%%
\begin{document}
	\maketitle

	Diese Einweisung erkl{\"a}rt die Handhabung der Dampfphasenl{\"o}tanlage \textit{VaporPhase One}, die zu beachtenden Regeln, Gefahren und Sicherheitshinweise.
	
	\english{This instruction explains use of the vapor phase soldering unit \textit{VaporPhase One}, and the associated rules, hazards and safety advice.}
	
	
	\section{Einleitung: Warum Dampfphasenl{\"o}ten?}
	
	Beim maschinellen L{\"o}ten mit L{\"o}tpaste ist es sehr wichtig, dass alle L{\"o}tstellen gleichm{\"a}{\ss}ig, gleichzeitig und kontrolliert entlang eines vom Hersteller der L{\"o}tpaste vorgegebenen Temperaturprofils erhitzt werden, um {\"u}berall auf der Platine gute L{\"o}tstellen gleichbleibend hoher Qualit{\"a}t zu erhalten. \\
	
	Man hat dabei die Wahl zwischen einer Anzahl von verschiedenen L{\"o}tverfahren, welche alle spezifische Vor- und Nachteile besitzen. Am h{\"a}ufigsten werden das Wellenl{\"o}ten und das Reflowl{\"o}ten verwendet. Beim Wellenl{\"o}ten wird der W{\"a}rmeenergieeintrag durch direkte Infrarotstrahlung erreicht, w{\"a}hrend beim Reflowl{\"o}ten nach dem Prinzip des Umluftbackofens die Luft um die Platine erhitzt und dadurch ein W{\"a}rmeeintrag erreicht wird.\\
	
	Das Wellenl{\"o}ten hat durch die direkte Einstrahlung den Vorteil, dass der W{\"a}rmeeintrag auf bestimmte Teile der Platine beschr{\"a}nkt werden kann, um bereits gel{\"o}tete oder w{\"a}rmeempfindliche Teile zu schonen. Dadurch kann aber ein W{\"a}rmeeintrag unter Bauteilen oder an Stellen die anderweitig abgeschattet sind wen {\"u}berhaupt nur durch Durcherhitzen der dar{\"u}berliegenden Teile erreicht werden, was zu ungleichm{\"a}{\ss}igen Ergebnissen und einer thermischen {\"u}berlastung der Bauteile f{\"u}hrt.\\
	
	Dieses Problem wird beim Reflowl{\"o}ten umgangen, indem man erhitze Luft als W{\"a}rme{\"u}bertragungsmedium verwendet. Dies verbessert zwar den W{\"a}rmeeintrag in abgeschatteten Stellen, aber durch die geringe W{\"a}rmekapazit{\"a}t der Luft ist ein gleichm{\"a}{\ss}iger W{\"a}rmeeintrag unter gr{\"o}{\ss}eren Bauteilen (z.B. BGA) weiterhin nur bedingt m{\"o}glich.\\
	
	Um diesem Problem zu begegnen, wird beim Dampfphasenl{\"o}ten statt erhitzter Luft ein spezielles W{\"a}rme{\"u}bertragungsmedium (\textit{Solvay Galden LS 230}) verwendet, welches bei der gew{\"u}nschten L{\"o}ttemperatur verdampft und eine deutlich h{\"o}here W{\"a}rmekapazit{\"a}t als Luft besitzt, was einen gleichm{\"a}{\ss}igen W{\"a}rmeeintrag auch unter gr{\"o}{\ss}eren Bauteilen erm{\"o}glicht. Es handelt sich dabei um einen Flourpolyether, eine Substanz die bei korrekter Anwendung sowohl gegen{\"u}ber elektronischen Platinen und Bauteilen als auch gegen{\"u}ber biologischen Systemen (z.B. Menschen) inert ist. Durch die Verwendung der Gasphase statt einer fl{\"u}ssigen Phase treten auch keine Probleme mit der Oberfl{\"a}chenspannung auf.
	
	\section{Sicherheitshinweise}
	
	\begin{itemize}
	
	\item \textbf{Verbrennungsgefahr am Wärmeübertragungsmedium} Jeglicher Kontakt mit dem Wärmeübertragungsmedium (gasförmig und flüssig) kann zu schweren Verbrennungen führen. Lötkammer erst bei einer Temperatur unter 70\textdegree C öffnen. Im Störungsfall (öffnen bei höherer Temperatur) Stecker ziehen, Abstand halten, und Betreuer rufen.

	\item \textbf{Verbrennungsgefahr an der frisch gelöteten Platine} Die Platine kann nach dem {\"o}ffnen der L{\"o}tkammer noch Temperaturen von {\"u}ber 100\textdegree C haben. Beim Entnehmen der Platine eine Zange oder ein anderes geeignetes w{\"a}rmefestes Hilfsmittel benutzen.
	\item \textbf{Rutschgefahr} Das W{\"a}rme{\"u}bertragungsmedium hat eine {\"o}lige Konsistenz und erzeugt sehr rutschige Oberfl{\"a}chen. Verschüttetes Wärmeübertragungsmedium sofort aufwischen.
	\item \textbf{Vergiftungsgefahr} Das W{\"a}rme{\"u}bertragungsmedium zersetzt sich oberhalb von ca 290\textdegree C in hochgiftige Flu{\ss}s{\"a}ure und kurzkettige flourorganische Verbindungen. Falls auf Grund einer Störung der Anlage die Temperatur des W{\"a}rme{\"u}bertragungsmediums 240\textdegree C {\"u}berschreitet, die Dampfphasenl{\"o}tanlage sofort am Hauptschalter oder durch ziehen des Netzsteckers abschalten und ein Betreuer zu rufen. Die L{\"o}tkammer nicht {\"o}ffnen.
	\item \textbf{Vergiftungsgefahr} Das W{\"a}rme{\"u}bertragungsmedium ist ein Gefahrstoff und als solcher zu behandeln. Anhalftende Tropfen beim Entnehmen der Platine durch absch{\"u}tteln oder abstreifen in die L{\"o}tkammer zur{\"u}ckzuf{\"u}hren. Nach dem L{\"o}ten gr{\"u}ndlich die H{\"a}nde waschen.
	\item \textbf{Überhitzungsgefahr} Die Lüftungsschlitze seitlich an der Lötanlage dürfen im Betrieb nicht verdeckt werden.
	\end{itemize}
	
	\section{Vorbereitung des L{\"o}tvorganges}
	\subsection{Geeignete Bauteile und Bauteilanordnungen}
	
	
	Grunds{\"a}tzlich ist das Dampfphasenl{\"o}ten nur f{\"u}r oberfl{\"a}chenmontierte Bauteile (SMD/SMT) geeignet. Bedrahtetet Bauteile (THT) m{\"u}ssen \textit{nach} dem L{\"o}ten in der Dapfphasenl{\"o}tanlage mit einer anderen Methode aufgel{\"o}tet werden, und d{\"u}rfen auch erst danach best{\"u}ckt werden, um eine unn{\"o}tige thermische Belastung oder ein Fallen der Bauteile in die L{\"o}tkammer zu verhindern.\\

Bauteile d{\"u}rfen nur einseitig best{\"u}ckt sein, da unten auf der Platine angebrachte Bauteile mit sehr hoher Wahrscheinlichkeit in das L{\"o}tbecken fallen. \\

Bauteile m{\"u}ssen au{\ss}erdem f{\"u}r die beim Dampfphasenl{\"o}ten auftretenden Temperaturen (230\textdegree C f{\"u}r 30 Sekunden) geeignet sein. Ob dies der Fall ist kann im Bauteiledatenblatt des Herstellers nachgesehen werden. Insbesondere bei Bauteilen mit thermoplastischen Anteilen besteht sonst die Gefahr, dass geschmolzener thermoplastischer Kunststoff in die L{\"o}tkammer tropft und das W{\"a}rme{\"u}bertragungsmedium kontaminiert.


\subsection{Best{\"u}cken}

Vor dem L{\"o}ten ist die Platine wie bei anderen automatisierten L{\"o}tverfahren auch mit L{\"o}tpaste zu versehen und zu best{\"u}cken. Die Bauteile etwas andr{\"u}cken, um die Haftwirkung der L{\"o}tpaste gut auszunutzen und die Gefahr des Verrutschens der Bauteile beim Verfahren des Lochbleches in der L{\"o}tkammer zu minimieren.

\section{Vorbereitung der L{\"o}tanlage}
\subsection{Vor dem Einschalten}

Vor dem Einschalten ist der K{\"u}hlwasserstand hinten Ger{\"a}t zu {\"u}berpr{\"u}fen. Falls er sich nicht zwischen den beiden Markierungen befindet, darf das Ger{\"a}t nicht eingeschaltet werden. Einen Betreuer rufen, der das K{\"u}hlwasser nachf{\"u}llen wird.

%\todo{Foto K{\"u}hlwasserschlauch mit Hervorhebung der F{\"u}llstandsmarkierungen}

Falls das Ger{\"a}t schon eingechaltet ist, kann dieser Schritt {\"u}bersprungen werden.

\subsection{Nach dem Einschalten}

\paragraph{F{\"u}llstand des W{\"a}rme{\"u}bertragungmediums {\"u}berpr{\"u}fen}

Nachdem das Ger{\"a}t eingeschalten wurde, \textit{und vor jedem L{\"o}tvorgang} ist zun{\"a}chst die L{\"o}tkammer zu {\"o}ffnen, und der F{\"u}llstand des W{\"a}rme{\"u}bertragungsmediums zu {\"u}berpr{\"u}fen. Die F{\"u}llstandsmarkierungen befinden sich in der L{\"o}tkammer unten links auf einem vertikalen Metallstreifen. Falls sich der F{\"u}llstand nicht zwischen den beiden Markierungen befindet darf das Ger{\"a}t nicht verwendet werden. Einen Betreuer rufen.

%\todo{Foto F{\"u}llstandsmarkierung in der L{\"o}tkammer mit Hervorhebung}

\paragraph{Einstellungen {\"u}berpr{\"u}fen}

Auf dem Bildschirm des Ger{\"a}tes durch zweimaliges Wischen nach links auf den rechten Bildschirm schalten. Dort sind folgende Einstellungen vorzunehmen bzw zu {\"u}berpr{\"u}fen:\\

Die SD-Karte des Ger{\"a}tes muss gemountet sein. Falls neben dem Wort \textit{SD-Card} ein Knopf \vpoMount zu sehen ist, ist dieser zu bet{\"a}tgen. Wenn dort stattdessen ein Knopf \vpoEject zu sehen ist, ist die SD-Karte korrekt gemountet.\\

Der Betriebsmodus ist auf \vpoEco einzustellen. Ein Betrieb in den Modi \vpoStandard oder \vpoFast ist verboten, da durch das fr{\"u}here {\"o}ffnen der L{\"o}tkammer eine zu gro{\ss}e Menge des W{\"a}rme{\"u}bertragungsmediums verloren geht. (1L des W{\"a}rme{\"u}bertragungsmediums kostet ca. 200 Euro!)\\

Das Profil \textit{custom\_230.csv} ist auszuw{\"a}hlen und mit dem Knopf \vpoSet zu best{\"a}tigen.\\
%\todo{Foto des Bildschirms mit den entsprechenden Einstellungen hervorgehoben}
\paragraph{Temperatursensorposition}

Der Temperatursensor (verdrillte gr{\"u}ne und wei{\ss}e Kabel) muss am Lochblech angebracht und au{\ss}erhalb oben and der R{\"u}ckseite des Ger{\"a}tes eingesteckt sein.

%\todo{Bild Temperatursensor innen und au{\ss}en}

\section{L{\"o}ten}

Nachdem alle Vorbereitungen abgeschlossen sind, kann die best{\"u}ckte Platine auf das Lochblech gelegt und die L{\"o}tkammer geschlossen werden.\\
Nach dem schlie{\ss}en visuell {\"u}berpr{\"u}fen, dass beim Schlie{\ss}en weder die Platine noch die Bauteile darauf verrutscht sind.\\
Danach kann der L{\"o}tvorgang auf dem linken Bildschirm gestartet werden.\\
%\todo{Bild linker Bildschirm, hervorhebung kn{\"o}pfe schlie{\ss}en und start}
Nach dem Start auf den mittleren Bildschirm schalten, und den L{\"o}tvorgang {\"u}berwachen. Dabei ist auf die m{\"o}glichst genaue {\"u}bereinstimmung der Ist-Temperatur (blaue Kurve) mit der Solltemperatur (rote Kurve), sowie auf die Temperatur des W{\"a}rme{\"u}bertragungsmediums zu achten, welche keinesfalls {\"u}ber 240\textdegree C steigen darf. Weiterhin ist die Temperatur der Platine (PCB) zu beobachten, diese muss evenfalls deutlich über die Umgebungstemperatur steigen. Falls eine der Temperaturen nicht steigt, oder andere nicht plausible Werte Zeigt (z.B. 0\textdegree C oder Umgebungstemperatur), ist die Anlage am Hauptschalter auszuschalten und ein Betreuer zu rufen.



%\todo{Bild mittlerer Bildschirm, hervorhebungen Kurven und Temperatur Galden}

Sofern keine St{\"o}rungen auftreten, l{\"a}uft der L{\"o}tvorgang automatisch ab. Nach dem L{\"o}tvorgang wird ein Abk{\"u}hlen der L{\"o}tkammer auf ca. 70\textdegree C abgewartet bevor die L{\"o}tkammer mit dem Ende des Programms automatisch {\"o}ffnet. Das Abk{\"u}hlen ist unbedingt abzuwarten.\\

Falls das Programm vorzeitig abgebrochen wird, ist ein {\"o}ffnen der L{\"o}tkammer oberhalb einer Temperatur des W{\"a}rme{\"u}bertragungsmediums von 70\textdegree C nur nach R{\"u}cksprache mit einem Betreuer und in Ausnahmef{\"a}llen erlaubt, da dadurch ein erh{\"o}hter Austritt noch gasf{\"o}rmigen W{\"a}rme{\"u}bertragungsmediums m{\"o}glich ist. Auch hier ist stattdessen die Temperatur des W{\"a}rme{\"u}bertragungsmediums am mittleren Bildschirm zu beobachten, und das Fallen der Temperatur unter 70\textdegree C abzuwarten.\\

Nach dem {\"o}ffnen der L{\"o}tkammer kann die Platine mit einer Zange (vorsicht hei{\ss}) entnommen werden. Eventuell noch anhaftende Tropfen des W{\"a}rme{\"u}bertragungsmediums sind in das L{\"o}tbecken abzusch{\"u}tteln oder abzustreifen bevor die Platine entg{\"u}ltig entnommen wird.\\

Nach dem L{\"o}ten unbedingt die H{\"a}nde waschen, um eine unebabsichtige Aufnahme des W{\"a}rme{\"u}bertragungsmediums in den K{\"o}rper zu vermeiden.

\section{Nach dem L{\"o}ten}

Nach dem L{\"o}ten ist kurz visuell zu {\"u}berpr{\"u}fen, ob keine Fremdk{\"o}rper in die L{\"o}tkammer gefallen sind. Danach ist die L{\"o}tkammer zu schlie{\ss}en. Wenn keine weiteren L{\"o}tvorg{\"a}nge geplant sind, ist das Ger{\"a}t am Hauptschalter auszuschalten.
	\newpage
	\section{Checkliste}

	\begin{itemize}

	\item Sind alle verwendeten Bauteile f{\"u}r das Reflow- oder Dampfphasenl{\"o}ten geeignet? (Temperaturbest{\"a}ndigkeit, sind keine Bauteile auf der Unterseite der Platine (auch schon vorher aufgel{\"o}tete)
	\item Ist der K{\"u}hlwasserstand innerhalb der beiden Markierungen? (wei{\ss}er Schlauch auf der R{\"u}ckseite) 
	\item Die Maschine kann jetzt am Hauptschalter (hinten rechts unten) eingeschaltet werden
	\item Ist die Temperatur des W{\"a}rme{\"u}bertragungsmediums unter 70\textdegree C? (Mittlerer Bildschirm, oben Mitte) Falls nicht, erst Abk{\"u}hlen abwarten.
	\item L{\"o}tkammer {\"o}ffnen (Linker Bildschirm, Knopf \vpoOpen)
	\item Ist der F{\"u}llstand des W{\"a}rme{\"u}bertragungsmediums zwischen den beiden Markierungen? (Kerben auf dem senkrechten Metallstreifen im L{\"o}tbecken unten links)
	\item Ist der Temperatursensor (verdrilltes gr{\"u}nes und wei{\ss}es Kabel) am Lochblech angebracht und auf H{\"o}he einer eigelegten Platine sowie am Ger{\"a}t hinten eingesteckt?
	\item Best{\"u}ckte Platine auf das Lochblech legen, Kammer schlie{\ss}en (Linker Bildschirm, Knopf \textit{Close})
	\item Ist die SD-Karte gemountet? Falls ein Knopf \vpoMount zu sehen ist, ist dieser zu bet{\"a}tigen. Ein Knopf \vpoEject zeigt an, dass die SD-Karte bereits gemountet ist. (Rechter Bildschirm)
	\item Ist der Modus \vpoEco ausgew{\"a}hlt? (Rechter Bildschirm)
	\item Ist das passende L{\"o}tprofil ausgew{\"a}hlt? Falls kein spezielles Profil ben{\"o}tig wird, das Profil \textit{custom\_230.csv} verwenden. (Rechter Bildschirm, nach dem Ausw{\"a}hlen den Knopf \vpoSet dr{\"u}cken)
	\item L{\"o}tvorgang starten (Linker Bildschirm, Knopf \textit{Start Reflow})
	\item Den L{\"o}tvorgang auf dem mittleren Bildschirm beobachten. Inbesondere ist die Temperatur des Galdens (im Bildschirm oben rechts) zu {\"u}berwachen, sie darf keinesfalls {\"u}ber 240\textdegree C steigen. Im Notfall sp{\"a}testens bei bei Erreichen und {\"u}berschreiten von 240\textdegree C das Ger{\"a}t sofort am Hauptschalter hinten rechts unten oder durch ziehen des Netzsteckers ausschalten und Betreuer rufen. 
	\item Auf das Ende des L{\"o}t- und Abk{\"u}hlvorganges warten (Temperatur des W{\"a}rme{\"u}bertragungsmediums unter 70\textdegree C). Die L{\"o}tkammer {\"o}ffnet automatisch. Vorsicht, die Platine hat immernoch eine Temperatur von ca. 100\textdegree C. Zum Entnehmen eine Zange verwendenden. Anhaftende Tropfen W{\"a}rme{\"u}bertragungsmedium an der Platine in die L{\"o}tkammer absch{\"u}tteln.
	\item Visuelle Kotrolle der L{\"o}tkammer auf hingefallene Fremdk{\"o}rper, dann L{\"o}tkammer schlie{\ss}en. 
	\item Zwischen mehreren L{\"o}tvorg{\"a}ngen ist die L{\"o}tkammer geschlossen zu halten, um die Verdunstung des W{\"a}rme{\"u}bertragungsmediums zu minimieren.
	\item Falls keine weiteren L{\"o}tvorg{\"a}nge geplant sind das Ger{\"a}t am Hauptschalter ausschalten. 
	
	\end{itemize}	
	
	\section{Wartung (f{\"u}r Betreuer)}
	
	\subsection{K{\"u}hlwasserstand}
	
	Sofern der K{\"u}hlwasserstand zu niedrig ist, einfach mit destilliertem/deionisiertem Wasser nachf{\"u}llen. Dann bitte Mail auf die Mailingliste, dass der K{\"u}hlwasserstand niedrig war, um die Dichigkeit der K{\"u}hlung zu {\"u}berwachen (damit wir mitbekommen, falls der K{\"u}hlwasserstand oft oder {\"o}fters nierdrig ist)
	
	 Das K{\"u}hlwasser in die Einf{\"u}ll{\"o}ffnung direkt {\"u}ber dem wei{\ss}en Schlauch einf{\"u}llen, nicht vergessen die Verschlusskappe wieder aufzuschrauben.
	
	\subsection{Galdenstand}	
	
	Galden kann einfach nachgef{\"u}llt werden, wenn der Stand niedrig ist, hier iat an sich nichts weiter zu beachten. 
	
	\subsection{Betriebsmedien ablassen}
	
	Falls die Betriebsmedien (Galden und Kühlwasser) abgelassen werden müssen, zum Beispiel zum Transport oder zur Reinigung der Lötkammer, ist folgendermaßen vorzugehen:
	
	\paragraph{Kühlwasser}
	Das Kühlwasser kann abgelassen werden, indem hinten am Gerät am weißen Kühlwaserschlauch die obere Klemme gelöst wird (bitte nicht verlieren!) und das Kühlwasser dann über den Schlauch abgelassen wird.
	
	\paragraph{Galden}
	Das Galden kann mit einer Spritze (in der blauen Zubehörkiste) aus dem Lötbecken in die Vorratsflasche transferriert werden. Bitte dabei möglichst alles transferieren, da Galden sehr teuer ist (1L kostet ca 200 Euro)
	
		\subsection{SD-Karte}
	Die SD-Karte befindet sich unter dem FabLab-Aufkleber rechts neben dem Display. Auf der SD-Karte sind die Lötprofile, und Logfiles. Bitte die Karte nur entnehmen wenn wirklich notwendig, und danach gleich wieder ins Gerät einlegen.
	
\vpoMount	
	
	\newpage
	\ccLicense{dampfphasenl{\"o}tanlage-einweisung}{Einweisung Dampfphasenl{\"o}tanlage}
	


\end{document}
